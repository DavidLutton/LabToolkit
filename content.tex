\section{Assessment 1 Segments}
U3 P2 Prepare a project specification\\
U3 P3 Prepare the procedures that will be followed when implementing the project\\
U3 P4 Use appropriate techniques to evaluate three potential solutions and select the best option for developement\\
U2 P7 Use appropriate ICT software package and hardware to present information\\
U3 P5 Outline the project solution and plans its implementation\\
.\\
U3 M2 Use a wide range of techniques and selection criteria to justify chosen solution\\
U2 M3 Evaluate the effectiveness of an ICT software package and its tools for preparation and presentation of information
.\\
U2 D2 Evaluate the use of an ICT presentation method and identify an alternative approach


\section{Initial Project Ideas}\textbf{2016/09/03}\\
I have been advised to for our initial ideas in a mind map\\
\includegraphics[width=\textwidth]{text_2_mind_map}

\section{Modus operandi}
Dates will be recorded in \gls{ISO8601} format\\
YYYY-MM-DD\\
The result of this every folder of name ordered files will be in date order\\
\\
Version control will be using the \gls{Git} version control system\\
\\
The specification MUST be defined by \gls{RFC2119}

\section{Engineering Project : Requirements}\textbf{2016/09/20}\\
The project consists or both project management and communications\\
This will require:\\
A logbook will all entries dated\\
Project specification\\
Logbook will itemise all communications:
\begin{itemize}
  \item Initial ideas and justification of project
  \item Specifications
  \item Technical drawings
  \item Schematic
  \item Calculations
  \item Component specification
  \item Simulation
  \item Planning
  \item Decisions
  \item Customer Communications
  \item Websites used
  \item Decision matrix for solutions
  \item SWOT analysis
  \item GANTT chart for planning / time-line
\end{itemize}


Have processes of
\begin{itemize}
  \item Specify
  \item Design
  \item Build
  \item Test
  \item Modify
  \item Evaluate
\end{itemize}

\section{Project : Selection Critical}\textbf{2016/09/27}\\
Should:\\
Provide something\\
Have design choices\\
Be achievable (could)\\
Possibly as I do in my workplace\\
Could use a project in work for project\\
 Where naturally accessible at work\\

\section{Project : Selection Criteria}\textbf{2016/10/04}\\
Could:\\
A physical product\\
A system product, a remote monitoring system\\
A service product, a engineering service\\
\\
Replication or extension of an existing product to enhance the service\\

\section{Project Selection}\textbf{2016/10/08}\\
I did some discovery work for a few of the project in my initial project ideas\\
IR Remote - nearly achieved before customer needs changed\\
Lidar has commercial modules available - down to timing accuracy\\
USB Type C PD specification is hundreds of pages long - too difficult\\

\section{Project Final Idea}\textbf{2016/10/11}\\
RF Immunity test system\\
\begin{itemize}
  \item Control test equipment
  \item Monitor and control via feedback path
  \item Produce test report data
  \item Use Calibration factors
  \item Apply test standard \gls{EN61000-4-6:2014}
\end{itemize}
Customers will be test engineers at work\\


\section{Needs analysis}\textbf{2016/10/18}\\
To access is there a commercial or technical market\\
A survey may be used to derive a set of design specifications\\
Product performs in a repeatable manner\\
Product uses proven technology\\
Low incremental running cost : per runtime\\
Simple to maintain\\
Adaptable to new tests\\
Must stop/abort a test in a safe manner\\
Must apply correction factors\\

Project Specification from customer:\\
Time scale\\
Constraint dates\\
Availability of test set up for development\\
Plan installation\\
Set up drawing / work instructions\\

\section{Select solution}\textbf{2016/10/25}\\
Select from three solutions
This business sector has a couple of ready made platform products\\
LabVIEW by NI\\
VEE Pro by Keysight\\
Python 3 by Python Foundation\\
  With addition of some modules to communicate with test equipment\\

I am familiar with Python\\
I am using Python with instruments  already\\
This project should be viable for the use case\\
My employer already owns relevant hardware\\
I have access to expertise and relevant experience\\

\section{Test process : Calculations for running test}\textbf{2016/11/01}\\
Calibrations are performed a higher amplitude than the test level\\
We need to calculate the level that we need to apply\\
Calibration\\

\begin{table}[h!]
\centering
\begin{tabular}{||c c c c||}
 \hline
 Frequency Hz & level V & siggen dBm & forward power dBm \\ [0.5ex]
 \hline
 150e3 & 18 & -9 & 38.64 \\
 200e3 & 18 & -9 & 38.46 \\
 \hline
\end{tabular}
\caption{Example Calibration data}
\label{table:1}
\end{table}

.\\
For a calibration performed at 18V\\
For a test run to be performed at 10V\\
Offset V = Calibration power V : Wanted power V\\
Offset dBm = V2dBm(offset dBm)\\
Target forward power = forward power dBm : Offset dBm\\

Levelling performed without and with any AM applied\\
Dwell for time in both CW and AM modulation?\\

\section{Estimated time}\textbf{2016/11/08}\\\lstinputlisting[language=Python, caption=Estimated Time]{py_estimatedtime.py}

\section{Unit conversion}\textbf{2016/11/08}\\
Power meters operate in units of Watts or dBm\\
Our target level for calibration is in Volts\\
Our meter has best linearity in Watts\\
Therefore conversion between Watts and Volts is required\\
\lstinputlisting[language=Python, caption=Power meter calculations]{py_powermetercalcs.py}

\section{Project Specification Revised}\textbf{2016/11/15}\\

Functional requirements when running a test\\
Instrument control\\
  Power meter\\
  Signal generator\\
Step frequency steps\\
Use/apply calibration factors\\
Calculate the expected power with given CDN factors at coupling point\\
Level with given tolerance to calculated power\\
Dwell time must be at-least minimum time\\
Display a plot during test\\
Have a mode of operation for validating an EUT fault\\
  mini sweep over last n points\\
Pause standby a test\\
Stop a test\\


\section{Project discussion with Stuart}\textbf{2016/11/22}\\
Discussed project with Stuart\\
Identified additional documentation requirements\:\\
Terminology\\
Test configurations\\
The installation and operation of the software MUST be repeatable for a given build\\
\begin{table}[h!]
\centering
\begin{tabular}{||c c c c c c c c||}
 \hline
 Criteria & Weighting & LabVIEW & VEEPro & Python & LabVIEW & VEEPro & Python \\ [0.5ex]
 \hline
 Costs & 5 & 2 & 4 & 8 & 10 &	20 & 40 \\
 Implementation ease & 3 & 4 & 1 & 7 & 12 & 3 & 21 \\
Machine tools available & 7 & 6 & 1 & 10  & 42 & 7 & 70 \\
Project deadline risk & 7 & 3 & 1 & 6 & 21 & 7 & 42 \\
Material availability & 5 & 6 & 2 & 10 & 30 & 10 & 50 \\
H \& S \& Risk assessment & 4 & 5 & 5 & 5 & 20 & 20 & 20 \\
Expertise to complete & 5 & 3 & 1 & 7 & 15 & 5 & 35 \\
Totals & 36 & . & . & . & 150 & 72 & 278 \\
\hline
\end{tabular}
\caption{Decision making table}
\label{table:4}
\end{table}
\textbf{2016/11/20}\\
\section{SWOT Analysis}\textbf{2016/11/27}\\
\begin{tikzpicture}[
    any/.style={minimum width=8cm,minimum height=8cm,%
                 text width=7.5cm,align=center,outer sep=0pt},
    header/.style={any,minimum height=1cm,fill=black!10},
    leftcol/.style={header,rotate=90},
    mycolor/.style={fill=#1, text=#1!60!black}
]

\matrix (SWOT) [matrix of nodes,nodes={any,anchor=center},%
                column sep=-\pgflinewidth,%
                row sep=-\pgflinewidth,%
                row 1/.style={nodes=header},%
                column 1/.style={nodes=leftcol},
                inner sep=0pt]
{
          &|[fill=helpful]| {\texta} & |[fill=harmful]| {\textb} \\
|[fill=internal]| {\textcn} & |[mycolor=S]| & |[mycolor=W]| \\
|[fill=external]| {\textdn} & |[mycolor=O]| & |[mycolor=T]| \\
};

\node[any, anchor=center] at (SWOT-2-2) {Project will build on existing experience\\Can reuse some existing code\\};
\node[any, anchor=center] at (SWOT-2-3) {Experence with control faults\\Experience with larger scale applications\\};
\node[any, anchor=center] at (SWOT-3-2) {Low operational expense\\ No Licensing expense from this project\\ };
\node[any, anchor=center] at (SWOT-3-3) {Intgeration with existing testing platform\\};

\end{tikzpicture}


\newpage
\section{Setup Diagrams}\textbf{2016/11/22}\\
Diagram showing Normal testing set-up and calibration set-up\\
\begin{table}[h!]
\centering
\begin{tabular}{||c c c c c||}
\hline
& & & & Supply \\
& & & & Pwr/Signal line \\
Signal Generator & RF Amplifier & Coupler & Atten & CDN \\
& & Cable/Atten & & Pwr/Signal line \\
& & Power Head & & EUT \\
& & Power Meter & & \\
\hline
\end{tabular}
\caption{Test configuration}
\label{table:3}
\end{table}

\begin{table}[h!]
\centering
\begin{tabular}{||c c c c c||}
\hline
& & & & 150 Ohms termination \\
Signal Generator & RF Amplifier & Coupler & Atten & CDN \\
& & Cable/Atten & & 100Ohms through \\
& & Power Head  & & 10dB Power Atten \\
& & Power Meter & & 10dB Atten \\
& & & & Calibration Power Head \\
& & & & Calibration Power Meter \\
\hline
\end{tabular}
\caption{Calibration configuration}
\label{table:4}
\end{table}


\section{Get data from a powermeter}\textbf{2016/12/6}\\
\lstinputlisting[language=Python, caption=Power meter data]{py_getpowermeterdata.py}

\section{Step frequency steps}\textbf{2016/12/13}\\
\lstinputlisting[language=Python, caption=Frequency Steps]{py_frequencysteps.py}

\section{TODO}
\subsection{Get / set data from a signal generator}
\subsection{Limit ramping up signal generator}
\subsection{Abort test}
\subsection{Pause off test}
\subsection{Rewind test - validate EUT fault}
\subsection{Display table during test - of last ten points}
\subsection{Display graph during test}
\subsection{Monitor line voltage during test}
\subsection{Exception handling}
Should store state and back trace when encountered unless deliberately handled\\

\newpage\section{About}\subsection{Python}
\label{sec:Python}
\gls{Python} is a programming language available from \href{https://python.org}{Python.org}\\

Windows XP support terminated with Python 3.4.4\\
Otherwise always use the highest numbered version\\

When installing Python offers to include the runtime in PATH always set this option on\\
Otherwise you need to invoke the Python runtime with the full path every time\\

\gls{venv} is a module that is part of the Python core that creates Python virtual environments\\
This venv can be used to install modules separately from the system installation\\
If used correctly this allows programs to be reliably installed and this provides reasonable assurance that the program will operate the same way on a different computer\\

Extra Python functionality is built up of modules available from \gls{Pypi} or \gls{Git}\\
These modules are installable via \gls{pip}\\

For example:\\
\`pip install pyserial\`\\
Installs the module to utilise serial ports and includes a small serial terminal application.\\

Python is a programming language that uses spaces to delineate nested flow control\\

I have written a number of .bat scripts to install, start, run, (Utilities, Serial console, GPIB console, Code checkers)\\
\input{AboutRFC2119}\subsection{Git}
\label{sec:Git}
Git is a fast version control system\\
Nearly all operations occur locally\\
Distributed, clones (replicas) are by default full copys of the project history\\
Users commit into a repository, repository’s have checksums for each file in a commit and respective metadata\\
Ergo modifying history will invalidate checksums in a provable way\\

Diffs for text based files or files that can be presented in a textual manner, can have diffs that represent the changes from the current master to the files in the working area\\

