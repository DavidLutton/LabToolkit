\documentclass[a4paper]{article}
\usepackage[utf8]{inputenc}

\usepackage{import}
%\usepackage{fontspec}
%\usepackage{polyglossia}
%\setmainfont[]{DejaVu Serif}

%\usepackage{showframe}

\usepackage{color}   %May be necessary if you want to color links
\usepackage{hyperref}
\hypersetup{
    colorlinks=true, %set true if you want colored links
    linktoc=all,     %set to all if you want both sections and subsections linked
    linkcolor=black,  %choose some color if you want links to stand out
    linktocpage=true
}
\usepackage[top=20mm, bottom=20mm, left=30mm, right=30mm]{geometry}
\usepackage[english]{babel}
\usepackage[utf8]{inputenc}
\usepackage{fancyhdr}

\pagestyle{fancy}
\fancyhf{}
\rhead{RF Immunity Test System}
\lhead{David Lutton}
\rfoot{Page \thepage}


\title{Project - RF Immunity Test System}
%\subtitle{}
\author{David Lutton}


\begin{document}
\maketitle

\tableofcontents
\end{tableofcontents}

\newpage
\section{Mode}
I will record dates in \href{https://en.wikipedia.org/wiki/ISO_8601}{ISO 8601} format\\
YYYY-MM-DD\\


\section{Initial Project Ideas - 2016/09/03}
I have been advised to for our initial ideas in a mind map\\
TODO: Attach IMAGE OF Mind map\\

\section{Engineering Project - Requirements - 2016/09/20}
The project consists or both project management and communications\\
This will require:\\
A logbook will all entrys dated\\
Project specification\\
\\
Logbook will itemise all comms:
\begin{itemize*}
  \item Initial ideas and justification of project
  \item Specifications
  \item Technical drawings
  \item schematic
  \item calculations
  \item component spec
  \item simulation
  \item Planning
  \item Decisions
  \item Customer Communications
  \item Websites used
  \item decision matrix for solutions
  \item SWOT analysis
  \item GANTT chart for planning / timeline
\end{itemize*}
\\
Have processes of:
\begin{itemize*}
  \item Specify
  \item design
  \item Build
  \item Test
  \item Modify
  \item Evaluate
\end{itemize*}

\section{Project - Selection Critical - 2016/09/27}
Should:\\
Provide something\\
Have design choices\\
be achievable (could)\\
Possibly as I do in my workplace\\
Could use a project in work for project\\
 Where naturally accessible at work\\

\section{Project - Selection Critical - 2016/10/04}
Could:\\
A physical product\\
A system product, a remote monitoring system\\
A service product, a engineering service\\
\\
Replication or extension of an existing product to enhance the service\\

\section{Project Selection}
TODO?\\

\newpage

\section{Project Final Idea - 2016/10/11}
RF Immunity test system\\
\begin{itemize*}
  \item Control test equipment
  \item Monitor and control via feedback path
  \item Produce test report data
  \item Use Calibration factors
  \item Apply test standard EN61000-4-6:2014
\end{itemize*}\\
Customers will be test engineers at work\\
TODO: embed TestSystemNetwork.xlsx  = \\
Diagram showing Normal testing setup and calibration setup\\

\section{Needs analysis - 2016/10/18}
To access is there a commercial or technical market\\
A survey may be used to derive a set of design specifications\\
product performs in a repeatable manner\\
product uses proven technology\\
low incremental running cost - per runtime\\
simple to maintain\\
adaptable to new tests\\
must stop/abort a test in a safe manner\\
must apply correction factors\\

Project spec from customer:\\
Time scale\\
constraint dates\\
availability of test set up for development\\
plan installation\\
set up drawing / work instructions\\

\section{Select solution 2016/10/25}
select from three solutions
This business sector has a couple of ready made platform products\\
LabVIEW by NI\\
VEE Pro by Keysight\\
Python 3 By Python Foundation\\
  With addition of some modules to communicate with test equipment\\

\section{Test process - Calculations 2016/11/01}
Calibrations are performed a higher amplitude than the test level we need to calculate the level that we need to apply\\
Calibration\\
TODO: TABLE\\
Frequency Hz, level V, siggen dBm, forward power dBm\\
\\
For a calibration performed at 18V\\
For a test run to be performed at 10V\\
Offset V = Cal power V - Wanted power V\\
Offset dBm = V2dBm(offset dBm)\\
Target forward power = forward power dBm - Offset dBm\\

Levelling performed without and with any AM applied\\
Dwell for time in both CW and AM modulation?\\

\section{Estimated time 2016/11/08}
TODO:

\section{Unit conversion 2016/11/08}
TODO:

\section{Project Specification Revised 2016/11/15}
TODO:

\newpage

\section{Terminology  /  Nomenclature}


\subsection{RF \& Testing}
\begin{itemize}
\item CDN: Coupling Decoupling Network - Interface in-between a power or signal line and injected RF
\item Hz = frequency
\item dBm = 10 * log10(power / 0.001) Power referenced to 1mW
\item EUT: Equipment under test
\item AM: Amplitude modulation \%80 - typical
\end{itemize}

\subsection{Software}
\begin{itemize}
\item \href{https://www.python.org/}{Python}: A programming language
\item \href{https://pypi.python.org/pypi}{pypi}: Python package index
\item \href{https://pypi.python.org/pypi/pip}{pip}: install modules from pypi
\item \href{https://docs.python.org/3/library/venv.html}{venv}: virtual environment for Python applications
\item Git: version control system \href{https://git-scm.com}{git-scm.com}
\end{itemize}




\end{document}
